\chapter{Lab01: Introduction}\label{apx:lab01}

This project is designed to introduce programing environment to students. It prints out \texttt{Hello\ world!} string and simulates a coin flip with an uneven probability. Additionally, it analyzes arrays of integers searching for specific value pairs.

\begin{itemize}
\item
  For the sake of this lab \textbf{stick to the GitLab browser UI}.
\item
  This assignment is completed when all tests are passing.
\end{itemize}

Before you proceed to instructions, make sure you have walked through the AMS Tutorial for Solvers. Always make sure there is \textbf{no pending merge request}!

\section{Instructions: lab part}\label{lab01-instructions-lab-part}

\begin{enumerate}
\def\labelenumi{\arabic{enumi}.}
\item
  Use \textbf{GitLab browser UI} editor.

  \begin{enumerate}
  \def\labelenumii{\arabic{enumii}.}
  \item
    Navigate to the source file in \texttt{./src/main/Introduction.java}.
  \item
    Take your time to explore the source file.
  \item
    Click Edit button to start editing.
  \end{enumerate}
\item
  Print ``Hello world!'' from the \texttt{main} method.

  \begin{enumerate}
  \def\labelenumii{\arabic{enumii}.}
  \item
    Replace the first \texttt{TODO} statement in \texttt{main} method with the following line of code. Use copy and paste.

\begin{verbatim}
System.out.println("Hello world!");
\end{verbatim}
  \item
    Double-check your modifications before committing by clicking on the Preview changes button in the editor's header.
  \item
    Describe what the current modification does into a commit message (you may copy and paste the current instruction) and press Commit changes.
  \item
    Make sure the \emph{first} test is passing. After pipeline stops, the test passed badge will show \texttt{1/4}.
  \end{enumerate}
\item
  Implement the \texttt{coinFlip} method.

  \begin{enumerate}
  \def\labelenumii{\arabic{enumii}.}
  \item
    Replace the \texttt{TODO} statement in \texttt{coinFlip} method with the following block of code. Use copy and paste and do not forget to remove the redundant empty return statement.

\begin{verbatim}
if (Math.random() < 0.55) {
  return "heads";
}
return "tails";
\end{verbatim}
  \item
    Double-check your modifications before committing by clicking on the Preview changes button in the editor's header.
  \item
    Describe what the current modification does into a commit message (you may copy and paste the current instruction) and press Commit changes.
  \item
    Make sure \emph{both first and second} tests are passing. After pipeline stops, the test passed badge will show \texttt{2/4}.
  \end{enumerate}
\item
  Evaluate 100 runs of \texttt{coinFlip} in the \texttt{main} method.

  \begin{enumerate}
  \def\labelenumii{\arabic{enumii}.}
  \item
    Replace the second \texttt{TODO} statement in \texttt{main} method with the following block of code. Use copy and paste.

\begin{verbatim}
int heads = 0;
for (int i = 0; i < 100; i++) {
  if (coinFlip() == "heads") {
    heads++;
  }
}
System.out.println(heads);
\end{verbatim}
  \item
    Double-check your modifications before committing by clicking on the Preview changes button in the editor's header.
  \item
    Describe what the current modification does into a commit message (you may copy and paste the current instruction) and press Commit changes.
  \item
    Make sure \emph{first three} tests are passing. After pipeline stops, the test passed badge will show \texttt{3/4}.
  \end{enumerate}
\end{enumerate}

\section{Instructions: assignment part}\label{lab01-instructions-assignment-part}

\begin{enumerate}
\def\labelenumi{\arabic{enumi}.}
\item
  Implement the \texttt{checkSum} method according to its Javadoc description.
\item
  Repeat the commit routine from above: double-check, describe and commit.
\item
  Make sure \emph{all tests} are passing. After pipeline stops, the test passed badge will show \texttt{4/4}.
\end{enumerate}

\section{Bonus questions}\label{bonus-questions}

\begin{enumerate}
\def\labelenumi{\arabic{enumi}.}
\item
  You may want to boost your coding style to 100\%. Can you figure out how?
\item
  Why is coding style important? See Google Java Style Guide and the internet.
\item
  What is the (expected) complexity of your \texttt{checkSum} implementation?
\item
  What is the best \texttt{checkSum} complexity you can think of? How about \emph{O(n)}?
\item
  Can you think of an \emph{O(n)} algorithm that would work for \emph{unsorted} arrays?
\end{enumerate}
