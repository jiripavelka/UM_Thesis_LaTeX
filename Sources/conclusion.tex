\chapter{Conclusion} \label{chap:conclusion}

A system has been successfully implemented and tested to provide full conventional assignment management for instructors (professors) solving all mentioned problems and issues both explicit and technical / legal.

AMS is fully documented both technically and in form of tutorials for instructors and for solvers. As demonstrated on two sample Java coding assignments and illustrated on several images, AMS is ready to be used for actual teaching.

\begin{itemize}
    \item \textbf{Unified and universal assignment management across courses.}\\
    Lack of coherence and integration leads to inefficiency.
    \item \textbf{Meets all legal requirements including ADA.}\\
    A big step closer to all students with or without disabilities.
    \item \textbf{Supports collective operations for professors.}\\
    Collective evaluation, export, similarity measure and more.
    \item \textbf{Provides current university education using best industry practices.}\\
    Brings competitive advantage for students and thus for the university.
    \item \textbf{Solves daily problems between students and instructors.}\\
    Both students and instructors can focus on education instead of dealing with technicalities.
\end{itemize}

AMS is 100\% ready to be deployed and used for Java Programming II course\footnote{CSC220, Computer Science department, University of Miami}. It is ready to be configured for specific (internal) environment and to cover multiple other courses regardless programming language or type of assignments. It is ready to help students, assistants and professors. \textbf{Ready to makes our lives easier.}
