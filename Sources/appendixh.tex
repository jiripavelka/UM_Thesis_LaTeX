\chapter{AMS Script User Manuals}\label{apx:ams-documentation}

\section{AMS 1 2021-02-16 GNU}\label{ams-1-2021-02-16-gnu-user-manuals}

\subsection{NAME}\label{name}

AMS - coding assignment handling

\subsection{SYNOPSIS}\label{synopsis}

\texttt{ams} {[}\texttt{-w}{]} \texttt{\textless{}command\textgreater{}} {[}\texttt{\textless{}args\textgreater{}}{]}

\subsection{DESCRIPTION}\label{description}

Provides commands to \emph{distribute} coding assignments to solvers, \emph{evaluate} individual solutions and \emph{measure} similarities between each other.

\subsection{OPTIONS}\label{options}

\begin{itemize}
\item
  \texttt{-w}, \texttt{-\/-working-dir\ WORKING\_DIR}

  \begin{itemize}
  \item
    Run as if \texttt{\$\{SCRIPT\}} was started in \texttt{WORKING\_DIR} instead of the current working directory.
  \end{itemize}
\end{itemize}

\subsection{COMMANDS}\label{commands}

These are common ams commands used in various situations:

\begin{itemize}
\item
  \texttt{collect}: Download repositories and evaluate against source tests.
\item
  \texttt{distribute}: Create or update user repositories with files from source project.
\item
  \texttt{evaluate}: Verify Java project and generate badges for README.
\item
  \texttt{help}: Display this usage.
\item
  \texttt{measure}: Compute software similarities between projects using Moss script.
\end{itemize}

\subsection{EXIT STATUS}\label{exit-status}

\begin{itemize}
\item
  1: Other error.
\end{itemize}

\subsection{AUTHOR}\label{author}

George J. Pavelka george@internetguru.io

\subsection{SEE ALSO}\label{see-also}

\texttt{ams-collect}(1), \texttt{ams-distribute}(1), \texttt{ams-evaluate}(1), \texttt{ams-measure}(1)

\section{AMS-COLLECT 1 2021-03-01 GNU}\label{ams-collect-1-2021-03-01-gnu-user-manuals}

\subsection{NAME}\label{name-1}

AMS-Collect - evaluate repositories against source tests

\subsection{SYNOPSIS}\label{synopsis-1}

\texttt{ams\ collect} {[}\texttt{-hdenops}{]}

\subsection{DESCRIPTION}\label{description-1}

Downloads all repositories from a \texttt{NAMESPACE} matching given \texttt{PREFIX}. Runs \texttt{ams\ evaluate} on working project in \texttt{WORKING\_DIR} replacing only editable files from individual solution repositories.

\subsection{OPTIONS}\label{options-1}

\begin{itemize}
\item
  \texttt{-h}, \texttt{-\/-help}

  \begin{itemize}
  \item
    Display usage.
  \end{itemize}
\item
  \texttt{-d}, \texttt{-\/-deadline\ DATE}

  \begin{itemize}
  \item
    Checkout each repository to the latest commit \emph{pushed} before \texttt{DATE}. \texttt{DATE} must be in ISO 8601 format,  \\e.g.~\texttt{2021-03-05} or \texttt{2021-03-05T00:41:21+00:00}. Uses latest existing commits by default.
  \end{itemize}
\item
  \texttt{-e}, \texttt{-\/-editable\ FILE...}

  \begin{itemize}
  \item
    In the source project, replace specific \texttt{FILE(s)} from each repository preserving other files that are not supposed to be edited. Default value is \texttt{src/main/*.java}.
  \end{itemize}
\item
  \texttt{-n}, \texttt{-\/-dry-run}

  \begin{itemize}
  \item
    Only process option validation. Would not proceed with cloning repositories and evaluation.
  \end{itemize}
\item
  \texttt{-o}, \texttt{-\/-output-dir\ DIR}

  \begin{itemize}
  \item
    Store collected repositories and results in \texttt{DIR}. Uses \texttt{mktemp} by default.
  \end{itemize}
\item
  \texttt{-p}, \texttt{-\/-prefix\ PREFIX}

  \begin{itemize}
  \item
    From given \texttt{NAMESPACE} (below) add only projects with matching \texttt{PREFIX} (empty by default).
  \end{itemize}
\item
  \texttt{-s}, \texttt{-\/-namespace\ NAMESPACE}

  \begin{itemize}
  \item
    Add projects from given \texttt{NAMESPACE}.
  \end{itemize}
\end{itemize}

\subsection{EXIT STATUS}\label{exit-status-1}

\begin{itemize}
\item
  1: Other error.
\item
  2: Invalid options or arguments.
\item
  3: Remote namespace (group) not found.
\end{itemize}

\subsection{AUTHOR}\label{author-1}

George J. Pavelka george@internetguru.io

\subsection{SEE ALSO}\label{see-also-1}

\texttt{ams}(1), \texttt{ams-distribute}(1), \texttt{ams-evaluate}(1), \texttt{ams-measure}(1)

\section{AMS-DISTRIBUTE 1 2021-02-14 GNU}\label{ams-distribute-1-2021-02-14-gnu-user-manuals}

\subsection{NAME}\label{name-2}

AMS-Distribute - distribute an assignment among users

\subsection{SYNOPSIS}\label{synopsis-2}

\texttt{ams\ distribute} {[}\texttt{-ahilnps}{]}

\subsection{DESCRIPTION}\label{description-2}

This script distributes files from \texttt{WORKING\_DIR} into \texttt{NAMESPACE/{[}PREFIX{]}USERNAME} for each \texttt{USERNAME} from stdin. For existing projects creates merge requests if files in \texttt{WORKING\_DIR} have changed.

\subsection{OPTIONS}\label{options-2}

\begin{itemize}
\item
  \texttt{-a}, \texttt{-\/-assign\ WHEN}

  \begin{itemize}
  \item
    Assign developer role to users for newly created projects and assign users to issues \texttt{always}, \texttt{never}, or according to account existence \texttt{auto} (default).
  \end{itemize}
\item
  \texttt{-h}, \texttt{-\/-help}

  \begin{itemize}
  \item
    Display usage.
  \end{itemize}
\item
  \texttt{-i}, \texttt{-\/-process-issues\ LABEL}

  \begin{itemize}
  \item
    Copy issues marked with \texttt{LABEL} into destination repositories. Note: \\ \texttt{WORKING\_DIR} must be a GitLab project.
  \end{itemize}
\item
  \texttt{-l}, \texttt{-\/-update-links}

  \begin{itemize}
  \item
    Replace all occurrences of the assignment project's remote URL and its current branch with destination repository remote URL including its main branch in \texttt{README.md}. Note: \texttt{WORKING\_DIR} must be a GitLab project.
  \end{itemize}
\item
  \texttt{-n}, \texttt{-\/-dry-run}

  \begin{itemize}
  \item
    Only process options and stdin validation. Would not proceed with create or update user repositories.
  \end{itemize}
\item
  \texttt{-o}, \texttt{-\/-output-dir\ DIR}

  \begin{itemize}
  \item
    Store distributed repositories in DIR. Uses \texttt{mktemp} by default.
  \end{itemize}
\item
  \texttt{-p}, \texttt{-\/-prefix\ PREFIX}

  \begin{itemize}
  \item
    Prepend PREFIX in front of created repositories names. PREFIX is empty by default.
  \end{itemize}
\item
  \texttt{-s}, \texttt{-\/-namespace\ NAMESPACE}

  \begin{itemize}
  \item
    Distribute repositories into NAMESPACE instead of the current project's namespace.
  \end{itemize}
\end{itemize}

\subsection{EXIT STATUS}\label{exit-status-2}

\begin{itemize}
\item
  1: Other error.
\item
  2: Invalid options or arguments including empty or missing stdin.
\item
  3: Some (or all) invalid users.
\end{itemize}

\subsection{EXAMPLES}\label{examples-2}

\begin{verbatim}
echo "solver1 solver2 solver3" | ams distribute -l -p "lab01-" \ 
-s "umiami/george/csc220/fall20"
\end{verbatim}

Given example distributes current directory into the following locations. The \texttt{-l} option updates links in README files.

\begin{verbatim}
umiami/george/csc220/fall20/lab01-solver1
umiami/george/csc220/fall20/lab01-solver2
umiami/george/csc220/fall20/lab01-solver3
\end{verbatim}

The following example does the same job dynamically. Assuming you are in a \texttt{lab01} folder on branch \texttt{fall20}.

\begin{verbatim}
grep 'AMS_USERS:' .gitlab-ci.yml | cut -d'"' -f2 | ams distribute -l \
-p "$(basename $PWD)-" -s "umiami/george/csc220/$(git rev-parse \
--abbrev-ref HEAD)"
\end{verbatim}

Calling the command dynamically is a number one prevention from accidentally distributing a different lab or even a working solution to all solvers. Different branch would create a separate namespace and different folder (lab) would distribute the assignment into appropriate lab repositories.

\subsection{AUTHOR}\label{author-2}

George J. Pavelka george@internetguru.io

\subsection{SEE ALSO}\label{see-also-2}

\texttt{ams}(1), \texttt{ams-collect}(1), \texttt{ams-evaluate}(1), \texttt{ams-measure}(1)

\section{AMS-EVALUATE 1 2021-02-14 GNU}\label{ams-evaluate-1-2021-02-14-gnu-user-manuals}

\subsection{NAME}\label{name-3}

AMS-Evaluate - evaluate project and create status badges

\subsection{SYNOPSIS}\label{synopsis-3}

\texttt{ams\ evaluate}

\subsection{DESCRIPTION}\label{description-3}

Evaluates current project on compilation, coding style and automatic tests. Generates status files and colored summary badges. Stores log files to be displayed and linked, e.g.~from a \texttt{README} file.

\subsection{EXIT STATUS}\label{exit-status-3}

\begin{itemize}
\item
  1: Other error.
\end{itemize}

\subsection{AUTHOR}\label{author-3}

George J. Pavelka george@internetguru.io

\subsection{SEE ALSO}\label{see-also-3}

\texttt{ams}(1), \texttt{ams-collect}(1), \texttt{ams-distribute}(1), \texttt{ams-measure}(1)

\section{AMS-MEASURE 1 2021-02-14 GNU}\label{ams-measure-1-2021-02-14-gnu-user-manuals}

\subsection{NAME}\label{name-4}

AMS-Measure - measure software similarities using Moss script

\subsection{SYNOPSIS}\label{synopsis-4}

\texttt{ams\ measure} {[}\texttt{-hinps}{]}

\subsection{DESCRIPTION}\label{description-4}

Downloads all repositories from a \texttt{NAMESPACE} matching given \texttt{PREFIX} and measures software similarities between each other. Uses an assignment project in \texttt{WORKING\_DIR} as a base file (see Moss documentation). The link with evaluation results appears at the end of the script.

\subsection{OPTIONS}\label{options-4}

\begin{itemize}
\item
  \texttt{-h}, \texttt{-\/-help}

  \begin{itemize}
  \item
    Display usage.
  \end{itemize}
\item
  \texttt{-i}, \texttt{-\/-ignore}

  \begin{itemize}
  \item
    Ignore non-existing namespaces (EXIT STATUS 3).
  \end{itemize}
\item
  \texttt{-n}, \texttt{-\/-dry-run}

  \begin{itemize}
  \item
    Only process option validation. Would not proceed with cloning repositories and Moss execution.
  \end{itemize}
\item
  \texttt{-p}, \texttt{-\/-prefix\ PREFIX}

  \begin{itemize}
  \item
    From given \texttt{NAMESPACE} (below) add only projects with matching \texttt{PREFIX} (empty by default).
  \end{itemize}
\item
  \texttt{-s}, \texttt{-\/-namespace\ NAMESPACE}

  \begin{itemize}
  \item
    Add projects from given \texttt{NAMESPACE}. May contain multiple values separated by space. Process all branches in project's namespace if missing or empty.
  \end{itemize}
\item
  \texttt{-o}, \texttt{-\/-output-dir\ DIR}

  \begin{itemize}
  \item
    Store measured repositories in DIR. Uses \texttt{mktemp} by default.
  \end{itemize}
\end{itemize}

\subsection{EXIT STATUS}\label{exit-status-4}

\begin{itemize}
\item
  1: Other error.
\item
  2: Invalid options or arguments.
\item
  3: Remote namespace (group) not found.
\end{itemize}

\subsection{AUTHOR}\label{author-4}

George J. Pavelka george@internetguru.io

\subsection{SEE ALSO}\label{see-also-4}

\texttt{ams}(1), \texttt{ams-collect}(1), \texttt{ams-distribute}(1), \texttt{ams-evaluate}(1)
